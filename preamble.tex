\usepackage{titlesec}

% \usepackage{times}
% \usepackage[usenglish]{babel}
\usepackage{subcaption}
\usepackage{graphicx}
\usepackage[onehalfspacing]{setspace} %used for footnote spacing

% symbols need to be loaded before unicode-math
\usepackage{amsmath}
\usepackage{amssymb}
\usepackage[ruled, linesnumbered, noend, vlined]{algorithm2e}
% \usepackage[noEnd=true,indLines=true]{algpseudocodex}
% \usepackage[ruled]{algorithm}
\usepackage{siunitx}
\sisetup{
group-separator = {,},
round-mode = places,
round-precision = 0,
group-minimum-digits = 3
}

\usepackage[bold-style=ISO]{unicode-math}
%----------------------------------------------------------------------------------------
%	Margins
%----------------------------------------------------------------------------------------
\geometry{
	paper=letterpaper, % Change to letterpaper for US letter
	inner=2.5cm, % Inner margin
	outer=2.5cm, % Outer margin
	% bindingoffset=.5cm, % Binding offset
	top=2.5cm, % Top margin
	bottom=2.5cm, % Bottom margin
	%showframe, % Uncomment to show how the type block is set on the page
}

\usepackage[sort,square,numbers]{natbib}
%%%\usepackage[sectionbib]{chapterbib}   %used for placing
%%%bibliography after each chapter
% \usepackage{bibentry}
\usepackage{fontspec}
\setmainfont{Times New Roman}



%----------------------------------------------------------------------------------------
%	Packages
%----------------------------------------------------------------------------------------
% \usepackage[hide]{chato-notes} % Custom note boxes
\usepackage[textwidth=1in, textsize=scriptsize]{todonotes}
\usepackage{booktabs} % Better tables
\usepackage{multirow}
\usepackage{xspace} % No need for "\" after macros
\usepackage{graphicx}
\usepackage{subcaption}
\usepackage{flafter} % Float figures and tables after sections
\usepackage{amsthm}
\usepackage[shortcuts]{extdash}

%% Customize lists,
% inline lists with enumerate*
% shortlabels lets you do \begin{enumerate}[(1)] for easy label formatting
\usepackage[inline,shortlabels]{enumitem}


%% subcaptions
% Make subfigure labels look like Figure~1.1(b)
\usepackage[labelformat=simple]{subcaption}
\renewcommand\thesubfigure{(\alph{subfigure})}
%% making \caption, \caption[]{} from caption package
% \renewcommand{\caption}[1]{\caption[]{#1}}

%----------------------------------------------------------------------------------------
% Niceties, writing modern laTex
%----------------------------------------------------------------------------------------
\usepackage[style=english]{csquotes} %uncomment for super fancy quotes
\SetBlockEnvironment{quotation}% So block quotes have indented paragraphs.

%----------------------------------------------------------------------------------------
% UMD TOC
%----------------------------------------------------------------------------------------
\usepackage{tocloft,calc}
\renewcommand{\cftlottitlefont}{\hspace*{\fill}\large}
\renewcommand{\cftafterlottitle}{\hspace*{\fill}}
\renewcommand{\cftloftitlefont}{\hspace*{\fill}\large}
\renewcommand{\cftafterloftitle}{\hspace*{\fill}}
\renewcommand{\cftchapaftersnum}{:\ }
\renewcommand{\cftchappresnum}{\chaptername\space}
\setlength{\cftchapnumwidth}{\widthof{\textbf{Appendix\ }}}
\makeatletter
\g@addto@macro\appendix{%
  \addtocontents{toc}{%
    \protect\renewcommand{\protect\cftchappresnum}{\appendixname\space}%
  }%
}
\setlength{\cftchapnumwidth}{9em} %adds space between chapter/appendix and title in toc


%---
% rust code listing
%---
\usepackage{color}
\usepackage{listings}
\definecolor{GrayCodeBlock}{RGB}{241,241,241}
\definecolor{BlackText}{RGB}{110,107,94}
\definecolor{RedTypename}{RGB}{182,86,17}
\definecolor{GreenString}{RGB}{96,172,57}
\definecolor{PurpleKeyword}{RGB}{184,84,212}
\definecolor{GrayComment}{RGB}{170,170,170}
\definecolor{GoldDocumentation}{RGB}{180,165,45}
\lstdefinelanguage{Rust}
{
	numbers=left,
	numberstyle=\small\ttfamily\color{GrayComment},
    columns=fullflexible,
    keepspaces=true,
	firstnumber=last,
    frame=single,
    framesep=0pt,
    framerule=0pt,
    framexleftmargin=4pt,
    framexrightmargin=4pt,
    framextopmargin=5pt,
    framexbottommargin=3pt,
    xleftmargin=30pt, % margin to indent code blocks
    xrightmargin=4pt,
    backgroundcolor=\color{GrayCodeBlock},
    basicstyle=\small\ttfamily\color{BlackText},
    keywords={
        true,false,
        unsafe,async,await,move,
        use,pub,crate,super,self,mod,
        struct,enum,fn,const,static,let,mut,ref,type,impl,dyn,trait,where,as,
        break,continue,if,else,while,for,loop,match,return,yield,in
    },
    keywordstyle=\color{PurpleKeyword},
    ndkeywords={
        bool,u8,u16,u32,u64,u128,i8,i16,i32,i64,i128,char,str,
        Self,Option,Some,None,Result,Ok,Err,String,Box,Vec,Rc,Arc,Cell,RefCell,HashMap,BTreeMap,
        macro_rules
    },
    ndkeywordstyle=\color{RedTypename},
    comment=[l][\color{GrayComment}\slshape]{//},
    morecomment=[s][\color{GrayComment}\slshape]{/*}{*/},
    morecomment=[l][\color{GoldDocumentation}\slshape]{///},
    morecomment=[s][\color{GoldDocumentation}\slshape]{/*!}{*/},
    morecomment=[l][\color{GoldDocumentation}\slshape]{//!},
    morecomment=[s][\color{RedTypename}]{\#![}{]},
    morecomment=[s][\color{RedTypename}]{\#[}{]},
    stringstyle=\color{GreenString},
    string=[b]"
}

%----------------------------------------------------------------------------------------
% MUST BE LAST:	Linking and context-dependent references
% - myst be loaded last in order: bookmarks > hyperref > cleveref
% - cleverref lets you do \cref{<label>} for "Figure X", "Table X" etc.
% - use \Cref for start of sentences
%----------------------------------------------------------------------------------------
%hyperref
\usepackage[colorlinks=true,urlcolor=black,linkcolor=blue,citecolor=blue,debug=true]{hyperref}
\usepackage{bookmark}
\usepackage[capitalize,nameinlink]{cleveref}
\crefformat{equation}{#2(#1)#3}
\Crefformat{equation}{#2Equation~(#1)#3}


%----------------------------------------------------------------------------------------
% Other commands
%----------------------------------------------------------------------------------------

\newtheorem{lemma}{Lemma}[chapter]
\newtheorem{theorem}{Theorem}[chapter]
\newtheorem{problem}{Problem}[chapter]
\titleformat{\chapter}{\large}{\chaptername~\thechapter:}{1em}{}
\renewcommand{\paragraph}[1]{{\bfseries #1.}\hspace{1em}}